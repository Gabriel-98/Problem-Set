\documentclass{article}




%=======================================================================================================================
%================================================== IMPORT PACKAGES ====================================================
%=======================================================================================================================
\usepackage{geometry} % Package that allows easily modify the dimensions of the sheet
\usepackage[english]{babel} % Package to load the language, in this case, english
\usepackage{graphicx} % Package to add images to the document
\usepackage{enumitem} % Package to customize lists (enumerate, itemize, and description)
\usepackage{longtable} % Package to use tables that appear on more than one page
\usepackage{amsmath} % Package with additional functionalities and environments for handling mathematical notations




%=======================================================================================================================
%=============================================== GENERAL CONFIGURATION =================================================
%=======================================================================================================================
\geometry{a4paper, vmargin=20mm, hmargin=16mm} % Modify dimensions (sheet format and margins)
\setlength{\parindent}{0cm} % The indentation of each paragraph is 0
\setlength{\parskip}{-2mm} % The distance between paragraphs is reduced by 2 mm
\setlist[itemize,1]{label=\textbullet} % Sets the dot as the default symbol for the first level of lists




%=======================================================================================================================
%============================================== ENVIRONMENT DECLARATIONS ===============================================
%=======================================================================================================================

% Environment for grouping the sample test cases in a table
%   [header-font]: Sans Serif (sf)
%   [examples-main-font]: Typewriter (tt) - The characters have the same width
\newenvironment{EnglishTestCases}
{
	\vspace{-4mm}
	\begingroup
	\tt
	\renewcommand{\arraystretch}{1.5} % Change the vertical spacing of all rows in the table
	\begin{longtable}{| p{85mm} | p{85mm} |}
		\hline
		{\sf {\bfseries Input}} & {\sf {\bfseries Output}} \\ \hline
		\endfirsthead
}
{
	\end{longtable}
	\endgroup
}

% Environment for grouping constraints
% Constraints are added with the command \item
\newenvironment{ConstraintList}
{
	\vspace{-2mm}
	\begin{itemize}
}
{
	\end{itemize}
}




%=======================================================================================================================
%================================================ COMMAND DECLARATIONS =================================================
%=======================================================================================================================

% Print problem header (problem title, author, memory limit, time limit)
%   [syntax]: \englishProblemHeader{problem-title}{author}{memory-limit}{time-limit}
\newcommand{\englishProblemHeader}[4]{
	\begin{center} {\huge \sf \bfseries #1} \end{center}
	\vspace{-2mm}
	\centerline{\sf \bfseries Author: #2}
	\vspace{1mm}
	\centerline{\sf \bfseries Memory Limit: #3}
	\vspace{1mm}
	\centerline{\sf \bfseries Time Limit: #4}
	\vspace{10mm}
}

% Print a subtitle
%   [syntax]: \printSubtitle{subtitle}
\newcommand{\printSubtitle}[1]{
	\vspace{1mm}
	\begin{flushleft} {\Large \sf \bfseries #1 \newline} \end{flushleft}
	\vspace{-4mm}
}

% Add a test case (input, output) to the table of cases defined by the EnglishTestCases environment
%   [syntax]: \addTest{input}{output}
\newcommand{\addTest}[2]{
	#1 % Input
	&
	#2 % Output
	\\
	\hline
}

% Show an image horizontally centered
%   [syntax]: \image{image-path}{width}{height}{caption}
\newcommand{\image}[4]{
	\begin{figure}[h]
		\centering
		\includegraphics[width=#2, height=#3]{#1}
		\ifx&#4&\empty
			% show nothing
		\else
			\caption{#4}
		\fi
	\end{figure}
}








\begin{document} % Start of document content


%=======================================================================================================================
%=================================================== PROBLEM HEADER ====================================================
%=======================================================================================================================
\englishProblemHeader{Rectangular Sum}{Gabriel Gutiérrez Tamayo}{16 MB}{1.0 s}




%=======================================================================================================================
%==================================================== DESCRIPTION ======================================================
%=======================================================================================================================
In this challenge, you are given a triangular board of $ n $ rows. The first row has one block, and the following rows
have a block more than the previous row. All the blocks have the same size and are numbered as follows:
\newline

\image{images/figure-1-block-numbering.png}{4.6cm}{4cm}{}

First, you must find the biggest rectangular area inside the triangular board, and then calculate the value of $ S $
which corresponds to the sum of the values belonging to the area found. If there are several areas with the same size,
choose the area that maximizes the value of $ S $. For example, when $ n = 5 $:

\image{images/figure-2-example-n-equal-5.png}{4cm}{4.15cm}{}

The maximum rectangular area is $ (3 * 3) $, which is represented in the previous image.
\newline

$ S = 4 + 5 + 6 + 8 + 9 + 10 + 13 + 14 + 15 = 84 $
\newline

Remember that the area of a rectangle is the multiplication of the two sides of the rectangle.
\newline




%=======================================================================================================================
%======================================================= INPUT =========================================================
%=======================================================================================================================
\printSubtitle{Input}

The first line of input contains an integer $ t ~ (1 \le t \le 10^{5}) $ indicating the number of test cases that
follow, one for line. Each test case contains a positive integer $ n ~ (1 \le n \le 10^{11}) $ indicating the number of
rows.
\newline




%=======================================================================================================================
%======================================================= OUTPUT ========================================================
%=======================================================================================================================
\printSubtitle{Output}

For each test case, you should print a line containing {\tt Case \#x:\ y}, where $ x $ is the test case number (starting
from $ 1 $) and $ y $ is the sum obtained. Note that this value is very large, so print the result modulo
$ 10^{9} + 7 $.
\newline




%=======================================================================================================================
%====================================================== EXAMPLES =======================================================
%=======================================================================================================================
\printSubtitle{Examples}

\begin{EnglishTestCases}
	\addTest{
		6 \newline
		1 \newline
		2 \newline
		3 \newline
		4 \newline
		5 \newline
		14
	}
	{
		Case \#1:\ 1 \newline
		Case \#2:\ 5 \newline
		Case \#3:\ 16 \newline
		Case \#4:\ 42 \newline
		Case \#5:\ 84 \newline
		Case \#6:\ 3612
	}
\end{EnglishTestCases}


\end{document} % End of document content
