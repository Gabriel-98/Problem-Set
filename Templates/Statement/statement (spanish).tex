\documentclass{article}




%=======================================================================================================================
%================================================= IMPORTAR PAQUETES ===================================================
%=======================================================================================================================
\usepackage{geometry} % Paquete que permite modificar fácilmente las dimensiones de la hoja
\usepackage[spanish]{babel} % Paquete para cargar el idioma, en este caso el español
\usepackage{graphicx} % Paquete para añadir imagenes al documento
\usepackage{enumitem} % Paquete para personalizar las listas (enumerate, itemize y description)
\usepackage{longtable} % Paquete para usar tablas que aparecen en más de una página




%=======================================================================================================================
%=============================================== CONFIGURACIÓN GENERAL =================================================
%=======================================================================================================================
\geometry{a4paper, vmargin=20mm, hmargin=16mm} % Modifica las dimensiones (el formato de la hoja y los márgenes)
\setlength{\parindent}{0cm} % La sangría de cada párrafo tiene tamaño 0
\setlength{\parskip}{-2mm} % La distancia entre párrafos se reduce 2 mm
\setlist[itemize,1]{label=\textbullet} % Establece el punto como símbolo por defecto para el primer nivel de las listas




%=======================================================================================================================
%============================================== DECLARACIÓN DE ENTORNOS ================================================
%=======================================================================================================================

% Entorno para agrupar en una tabla los casos de prueba de ejemplo
%   [fuente-encabezado]: Sans Serif (sf)
%   [fuente-principal-ejemplos]: Typewriter (tt) - los caracteres tienen el mismo ancho
\newenvironment{SpanishTestCases}
{ 
	\vspace{-4mm}
	\begingroup
	\tt
	\renewcommand{\arraystretch}{1.5} % Cambia el espaciado vertical de todas las filas de la tabla
	\begin{longtable}{| p{85mm} | p{85mm} |}
		\hline
		{\sf {\bfseries Entrada}} & {\sf {\bfseries Salida}} \\ \hline
		\endfirsthead
}
{
	\end{longtable}
	\endgroup
}

% Entorno para agrupar restricciones
% Las restricciones se adicionan con el comando \item
\newenvironment{ConstraintList}
{
	\vspace{-2mm}
	\begin{itemize}
}
{
	\end{itemize}
}




%=======================================================================================================================
%============================================== DECLARACIÓN DE COMANDOS ================================================
%=======================================================================================================================

% Imprime el encabezado del problema (título del problema, nombre del autor, límite de memoria, límite de tiempo)
%   [sintaxis]: \spanishProblemHeader{titulo-del-problema}{autor}{limite-de-memoria}{limite-de-tiempo}
\newcommand{\spanishProblemHeader}[4]{
	\begin{center} {\huge \sf \bfseries #1} \end{center}
	\vspace{-2mm}
	\centerline{\sf \bfseries Autor: #2}
	\vspace{1mm}
	\centerline{\sf \bfseries Límite de memoria: #3}
	\vspace{1mm}
	\centerline{\sf \bfseries Límite de tiempo: #4}
	\vspace{10mm}
}

% Imprime un subtítulo
%   [sintaxis]: \printSubtitle{subtitulo}
\newcommand{\printSubtitle}[1]{
	\vspace{1mm}
	\begin{flushleft} {\Large \sf \bfseries #1 \newline} \end{flushleft}
	\vspace{-4mm}
}

% Agrega un caso de prueba (entrada, salida) a la tabla de casos definida por el entorno SpanishTestCases
%   [sintaxis]: \addTest{entrada}{salida}
\newcommand{\addTest}[2]{
	#1 % Entrada
	&
	#2 % Salida
	\\
	\hline
}

% Muestra una imagen centrada horizontalmente
%   [sintaxis]: \image{ruta-de-la-imagen}{ancho}{alto}{pie-de-imagen}
\newcommand{\image}[4]{
	\begin{figure}[h]
		\centering
		\includegraphics[width=#2, height=#3]{#1}
		\ifx&#4&\empty
			% show nothing
		\else
			\caption{#4}
		\fi
	\end{figure}
}








\begin{document} % Inicio del contenido del documento


%=======================================================================================================================
%============================================== ENCABEZADO DEL PROBLEMA ================================================
%=======================================================================================================================
\spanishProblemHeader{Título del problema}{Autor}{Memoria MB}{Tiempo s}




%=======================================================================================================================
%==================================================== DESCRIPCIÓN ======================================================
%=======================================================================================================================
Descripción del problema.
\newline

\image{images/figure-1.png}{9.6cm}{8cm}{Descripción de la figura}




%=======================================================================================================================
%====================================================== ENTRADA ========================================================
%=======================================================================================================================
\printSubtitle{Entrada}

Descripción de la entrada.
\newline




%=======================================================================================================================
%=================================================== RESTRICCIONES =====================================================
%=======================================================================================================================
\printSubtitle{Restricciones}

\begin{ConstraintList}
	\item $ R1 $
	\item $ R2 $
	\item $ R3 $
\end{ConstraintList}




%=======================================================================================================================
%======================================================= SALIDA ========================================================
%=======================================================================================================================
\printSubtitle{Salida}

Descripción de la salida.
\newline




%=======================================================================================================================
%====================================================== EJEMPLOS =======================================================
%=======================================================================================================================
\printSubtitle{Ejemplos}

\begin{SpanishTestCases}
    \addTest{
		Primera línea de la entrada \newline
		Segunda línea de la entrada \newline
		Tercera línea de la entrada
	}
	{
		Primera línea de la salida \newline
		Segunda línea de la salida
	}
\end{SpanishTestCases}


\end{document} % Fin del contenido del documento
